Цель работы:
научиться создавать и использовать классы в программах на языке программирования Java.

% = = = = = Task 1 = = = = =

\paragraph{Задание 1} \hspace{0pt}

\textbf{Реализовать простой класс}.

Требования к выполнению
\begin{enumerate}
    \item Реализовать пользовательский класс по варианту.
    \item Создать другой класс с методом main,
    в котором будут находится примеры использования пользовательского класса.
\end{enumerate}

Для каждого класса
\begin{enumerate}
    \item Создать поля классов
    \item Создать методы классов
    \item Добавьте необходимые get и set методы (по необходимости)
    \item Укажите соответствующие модификаторы видимости
    \item Добавьте конструкторы
    \item Переопределить методы toString() и equals()
\end{enumerate}

\paragraph{Вариант 5} \hspace{0pt}

\textbf{Множество целых чисел ограниченной мощности}
- Предусмотреть возможность объединения двух множеств,
вывода на печать элементов множества,
а так же метод, определяющий,
принадлежит ли указанное значение множеству.
Класс должен содержать методы,
позволяющие добавлять и удалять элемент в/из множества.
Конструктор должен позволить создавать объекты с начальной инициализацией.
Мощность множества задается при создании объекта.
\textbf{Реализацию множества осуществить на базе одномерного массива}.
Реализовать метод equals, выполняющий сравнение объектов данного типа.

\lstinputlisting[language=java]
{../src/task1_option5/src/com/example/Main.java}

\lstinputlisting[language=java]
{../src/task1_option5/src/com/example/intPower.java}

\lstinputlisting[language=java]
{../src/task1_option5/src/com/example/intPowerMain.java}

\newpage

\lstinputlisting[language=Out]
{_tex/task1_option5.txt}

\newpage

% = = = = = Task 2 = = = = =

\paragraph{Задание 2} \hspace{0pt}

Разработать автоматизированную систему на основе некоторой структуры данных,
манипулирующей объектами пользовательского класса.
Реализовать требуемые функции обработки данных

Требования к выполнению
\begin{enumerate}
    \item Задание посвящено написанию классов,
    решающих определенную задачу автоматизации;
    \item Данные для программы загружаются из файла (формат произволен).
    Файл создать и написать вручную.
\end{enumerate}

\paragraph{Вариант 5} \hspace{0pt}

Моделирование файловой системы

Составить программу, которая моделирует заполнение гибкого диска (1440 Кб). В процессе
работы файлы могут записываться на диск и удаляться с него.

С каждым файлом (File) ассоциированы следующие данные:

\begin{enumerate}
    \item Размер
    \item Расширение
    \item Имя файла
    \item Как файлы могут трактоваться и директории,
    которые в свою очередь содержат другие файлы и папки.
\end{enumerate}

\paragraph {Решение:} \hspace{0pt}

\lstinputlisting[]
{../src/task2_option5/files/file.json}

\lstinputlisting[language=c]
{../src/task2_option5/task2_option5/Program.cs}

\newpage

\lstinputlisting[language=Out]
{_tex/task2_option5.txt}
