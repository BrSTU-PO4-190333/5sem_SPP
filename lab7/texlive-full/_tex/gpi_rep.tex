Цель работы:
освоить возможности языка программирования Java в построении графических приложений.

% % = = = = =

\paragraph{Задание 1} \hspace{0cm}

\textbf{Построение графических примитивов и надписей}

Требования к выполнению
\begin{itemize}
    \item[-] Реализовать соответствующие классы, указанные в задании;
    \item[-] Организовать ввод параметров для создания объектов (можно использовать файлы);
    \item[-] Осуществить визуализацию графических примитивов, решить поставленную задачу
\end{itemize}

\textbf{Вариант 5}.
Изобразить в окне приложения (апплета) отрезок,
вращающийся в плоскости экрана вокруг одной из своих концевых точек.
Цвет прямой должен изменяться при переходе от одного положения к другому.

\lstinputlisting[language=java]
{../gpi_spp5_lab7_task1_option5/src/com/example/GPI_Main.java}

\lstinputlisting[language=java]
{../gpi_spp5_lab7_task1_option5/src/com/example/GPI_SimpleGUI.java}

% % = = = = =

\paragraph{Задание 2} \hspace{0cm}

\textbf{Реализовать построение заданного типа фрактала по варианту}.

Везде, где это необходимо, предусмотреть ввод параметров, влияющих на внешний вид фрактала.

\textbf{Вариант 5}.
Дерево Пифагора

\lstinputlisting[language=java]
{../gpi_spp5_lab7_task2_option5/src/com/example/GPI_Main.java}

\lstinputlisting[language=java]
{../gpi_spp5_lab7_task2_option5/src/com/example/GPI_SimpleGUI.java}

\lstinputlisting[language=java]
{../gpi_spp5_lab7_task2_option5/src/com/example/GRI_PythagoreanTree.java}

\lstinputlisting[language=java]
{../gpi_spp5_lab7_task2_option5/src/com/example/GPI_Point2D.java}
