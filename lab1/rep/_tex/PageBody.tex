Цель работы:
приобрести практические навыки обработки параметров командной строки,
закрепить базовые знания языка программирования Java при решении практических задач.

% = = = = = Task 1 = = = = =

\paragraph{Задание 1} \hspace{0pt}

Для переданной в качестве параметра последовательности из N целых чисел написать утилиту сфункционалом:
Вывод размаха последовательности (разницы между максимальным и минимальным числом).

% Выполнение:

\subparagraph{Код программы} \hspace{0pt}

\lstinputlisting[language=Java]
{../src/task1_option5/src/com/example/Main.java}

\subparagraph{Специфика ввода} \hspace{0pt}

> java com/example/Main <1-й элемент массива> <2-й элемент массива> <3-й элемент массива> <4-й элемент массива>

\subparagraph{Пример} \hspace{0pt}

> java com/example/Main 3 6 8 3 1 5 9

\subparagraph{Специфика вывода} \hspace{0pt}

min - max = <значение>

\subparagraph{Пример} \hspace{0pt}

min - max = 8

\subparagraph{Вывод:}
научился разрабатывать простейшие программы на ЯП Java,
получил практический опыт работы с компилятором javac.

% \newpage

% = = = = = Task 2 = = = = =

\paragraph{Задание 2} \hspace{0pt}

Написать функцию, выполняющую указанную операцию над массивом.
Использовать только базовые возможности языка, без привлечения специализированных функций для обработки коллекций.
Ввод массивов выполнять из командной строки.
Напишите метод long[] removeElement(long[] array, long element), который ищет и удаляетиз массива указанный элемент.

% Выполнение:

\subparagraph{Код программы} \hspace{0pt}

\lstinputlisting[language=Java]
{../src/task2_option5/src/com/example/Main.java}

\subparagraph{Специфика ввода} \hspace{0pt}

> java com/example/Main <1-й элемент массива> <2-й элемент массива> <3-й элемент массива> <4-й элемент массива>

\subparagraph{Пример} \hspace{0pt}

> java com/example/Main 2 2 1 2 3 2 4 2 5 6 2 2 2 7 2 2 2 2 8 2 2 2

\subparagraph{Специфика вывода} \hspace{0pt}

<1-й элемент массива> <2-й элемент массива> <3-й элемент массива> <4-й элемент массива>

\subparagraph{Пример} \hspace{0pt}

1 3 4 5 6 7 8

\subparagraph{Вывод:}
научился разрабатывать простейшие программы на ЯП Java,
получил практический опыт работы с компилятором javac.

% \newpage

% = = = = = Task 3 = = = = =

\paragraph{Задание 3} \hspace{0pt}

Написать функцию String randomString(int lenght, boolean asciiOnly)
для генерации случайных строк заданного размера.
Функция должна принимать флаг asciiOnly, определяющий,
должны ли в итоговой строке быть только ASCII символы.

% Выполнение:

\subparagraph{Код программы} \hspace{0pt}

\lstinputlisting[language=Java]
{../src/task3_option5/src/com/example/Main.java}

\subparagraph{Специфика ввода} \hspace{0pt}

> java com/example/Main <размер строки> [<символы только ASCII>]

\subparagraph{Пример} \hspace{0pt}

> java com/example/Main 8

> java com/example/Main 8 true

> java com/example/Main 8 false

\subparagraph{Специфика вывода} \hspace{0pt}

<строка>

\subparagraph{Пример} \hspace{0pt}

o@|S!S{y

\subparagraph{Вывод:}
научился разрабатывать простейшие программы на ЯП Java,
получил практический опыт работы с компилятором javac.
