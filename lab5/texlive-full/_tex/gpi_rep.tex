Цель работы:
приобрести практические навыки в области объектно-ориентированного проектирования.

% = = = = =

\paragraph{Задание 1} \hspace{0cm}

Реализовать абстрактные классы или интерфейсы,
а также наследование и полиморфизм для следующих классов:

\textbf{Вариант 5}.
interface Здание <--- abstract class Общественное Здание <--- class Театр.

\lstinputlisting[language=java]
{../gpi_spp5_lab5_task1_option5/src/com/example/Main.java}

\lstinputlisting[language=java]
{../gpi_spp5_lab5_task1_option5/src/com/example/Building.java}

\lstinputlisting[language=java]
{../gpi_spp5_lab5_task1_option5/src/com/example/PublicBuilding.java}

\lstinputlisting[language=java]
{../gpi_spp5_lab5_task1_option5/src/com/example/Theatre.java}

% = = = = =

\paragraph{Задание 2} \hspace{0cm}

В следующих заданиях требуется создать суперкласс (абстрактный класс, интерфейс)
и определить общие методы для данного класса.
Создать подклассы, в которых добавить специфические свойства и методы.
Часть методов переопределить.
Создать массив объектов суперкласса и заполнить объектами подклассов.
Объекты подклассов идентифицировать конструктором по имени или идентификационному номеру.
Использовать объекты подклассов для моделирования реальных ситуаций и объектов.

\textbf{Вариант 5}.
Создать абстрактный класс Работник фирмы и подклассы Менеджер,
Аналитик, Программист, Тестировщик, Дизайнер, Бухгалтер.
Реализовать логику начисления зарплаты.

\lstinputlisting[language=java]
{../gpi_spp5_lab5_task2_option5/src/com/example/Main.java}

\lstinputlisting[language=java]
{../gpi_spp5_lab5_task2_option5/src/com/example/Employee.java}

\lstinputlisting[language=java]
{../gpi_spp5_lab5_task2_option5/src/com/example/Manager.java}

\lstinputlisting[language=java]
{../gpi_spp5_lab5_task2_option5/src/com/example/Analyst.java}

\lstinputlisting[language=java]
{../gpi_spp5_lab5_task2_option5/src/com/example/Programmer.java}

\lstinputlisting[language=java]
{../gpi_spp5_lab5_task2_option5/src/com/example/Tester.java}

\lstinputlisting[language=java]
{../gpi_spp5_lab5_task2_option5/src/com/example/Designer.java}

\lstinputlisting[language=java]
{../gpi_spp5_lab5_task2_option5/src/com/example/Accountant.java}

% = = = = =

\paragraph{Задание 3} \hspace{0cm}

В задании 3 ЛР №4, где возможно,
заменить объявления суперклассов объявлениями абстрактных классов или интерфейсов.

\textbf{Вариант 5}.
Система \textbf{Библиотека}. \textbf{Читатель} оформляет \textbf{Заказ} на \textbf{Книгу}.
Система осуществляет поиск в \textbf{Каталоге}.
\textbf{Библиотекарь} выдает \textbf{Читателю Книгу} на абонемент или в читальный зал.
При невозвращении \textbf{Книги Читателем}
он может быть занесен \textbf{Администратором} в <<черный список>>.

\lstinputlisting[language=csh, name=gpi_spp5_lab5_task3_option5/Program.cs,]
{../gpi_spp5_lab5_task3_option5/gpi_spp5_lab5_task3_option5/Program.cs}

\newpage

\lstinputlisting[language=csh, name=gpi_spp5_lab5_task3_option5/gpi_AbstractLibrary.cs,]
{../gpi_spp5_lab5_task3_option5/gpi_spp5_lab5_task3_option5/gpi_AbstractLibrary.cs}

\newpage

\lstinputlisting[language=csh, name=gpi_spp5_lab5_task3_option5/gpi_Library__menus.cs,]
{../gpi_spp5_lab5_task3_option5/gpi_spp5_lab5_task3_option5/gpi_Library__menus.cs}

\lstinputlisting[language=csh, name=gpi_spp5_lab5_task3_option5/gpi_Library__books.cs,]
{../gpi_spp5_lab5_task3_option5/gpi_spp5_lab5_task3_option5/gpi_Library__books.cs}

\newpage

\lstinputlisting[language=csh, name=gpi_spp5_lab5_task3_option5/gpi_Library__reader.cs,]
{../gpi_spp5_lab5_task3_option5/gpi_spp5_lab5_task3_option5/gpi_Library__reader.cs}
